%%%%%%%%%%%%%%%%%%% vorlage.tex %%%%%%%%%%%%%%%%%%%%%%%%%%%%%
%
% LaTeX-Vorlage zur Erstellung von Projekt-Dokumentationen
% im Fachbereich Informatik der Hochschule Trier
%
% Basis: Vorlage svmono des Springer Verlags
%
%%%%%%%%%%%%%%%%%%%%%%%%%%%%%%%%%%%%%%%%%%%%%%%%%%%%%%%%%%%%%

\documentclass[envcountsame,envcountchap, deutsch]{i-studis}

\usepackage{makeidx}         	% Index
\usepackage{multicol}        	% Zweispaltiger Index
%\usepackage[bottom]{footmisc}	% Erzeugung von Fu�noten

%%-----------------------------------------------------
%\newif\ifpdf
%\ifx\pdfoutput\undefined
%\pdffalse
%\else
%\pdfoutput=1
%\pdftrue
%\fi
%%--------------------------------------------------------
%\ifpdf
%\usepackage[pdftex]{graphicx}
%\usepackage[pdftex,plainpages=false]{hyperref}
%\else
\usepackage{graphicx}
\usepackage[plainpages=false]{hyperref}
%\fi

%%-----------------------------------------------------
\usepackage{color}				% Farbverwaltung
%\usepackage{ngerman} 			% Neue deutsche Rechtsschreibung
\usepackage[english, ngerman]{babel}
%\usepackage[latin1]{inputenc} 	% Erm�glicht Umlaute-Darstellung
\usepackage[utf8]{inputenc}  	% Erm�glicht Umlaute-Darstellung unter Linux (je nach verwendetem Format)

%-----------------------------------------------------
\usepackage{listings} 			% Code-Darstellung
\lstset
{
	basicstyle=\scriptsize, 	% print whole listing small
	keywordstyle=\color{blue}\bfseries,
								% underlined bold black keywords
	identifierstyle=, 			% nothing happens
	commentstyle=\color{red}, 	% white comments
	stringstyle=\ttfamily, 		% typewriter type for strings
	showstringspaces=false, 	% no special string spaces
	framexleftmargin=7mm, 
	tabsize=3,
	showtabs=false,
	frame=single, 
	rulesepcolor=\color{blue},
	numbers=left,
	linewidth=146mm,
	xleftmargin=8mm
}
\usepackage{textcomp} 			% Celsius-Darstellung
\usepackage{amssymb,amsfonts,amstext,amsmath}	% Mathematische Symbole
\usepackage[german, ruled, vlined]{algorithm2e}
\usepackage[a4paper]{geometry} % Andere Formatierung
\usepackage{bibgerm}
\usepackage{array}
\hyphenation{Ele-men-tar-ob-jek-te  ab-ge-tas-tet Aus-wer-tung House-holder-Matrix Le-ast-Squa-res-Al-go-ri-th-men} 		% Weitere Silbentrennung bei Bedarf angeben
\setlength{\textheight}{1.1\textheight}
\pagestyle{myheadings} 			% Erzeugt selbstdefinierte Kopfzeile
\makeindex 						% Index-Erstellung

%--------------------------------------------------------------------------
\begin{document}
%------------------------- Titelblatt -------------------------------------
\title{Funktionsprinzip und Anwendungsbeispiele des Dijkstra-Algorithmus}
\project{Ausarbeitung zur Vorlesung Wissenschaftliches Arbeiten}
%--------------------------------------------------------------------------
\supervisor{Titel Vorname Name} 		% Betreuer der Arbeit
\author{Bearbeiter 1: Thomas Jürgensen \\Bearbeiter 2: Annika Kremer \\Bearbeiter 3: Tobias Meier}							% Autor der Arbeit
\groupid{13}
\address{Trier,} 							% Im Zusammenhang mit dem Datum wird hinter dem Ort ein Komma angegeben
\submitdate{\today} 				% Abgabedatum
%\begingroup
%  \renewcommand{\thepage}{title}
%  \mytitlepage
%  \newpage
%\endgroup
\begingroup
  \renewcommand{\thepage}{Titel}
  \mytitlepage
  \newpage
\endgroup
%--------------------------------------------------------------------------
\frontmatter 
%--------------------------------------------------------------------------
%\kurzfassung

%% deutsch
\paragraph*{}


Im Rahmen dieser Arbeit wird ein Gesamtüberblick über den Dijkstra-Algorithmus vermittelt.
Dazu wird zunächst auf Graphen eingegangen, da der Algorithmus auf Graphen ausgelegt ist. Es wird erklärt, wie Graphen repräsentiert werden können, um die Eingabe nachzuvollziehen.
Anschließend wird die Funktionsweise des Algorithmus erläutert und dessen Komplexität bezüglich Laufzeit  beschrieben. Zur Veranschaulichung wird eine mögliche Implementierung als Codebeispiel gezeigt.
Abschließend werden die vielfältigen Anwendungsbereiche beschrieben, um zu verdeutlichen, weshalb der Algorithmus unverzichtbarer Bestandteil des heutigen Zeitalters ist. \\ \\

In the course of this work, an overview of dijkstra's algorithm is given.
For this, the first discussed topic is graphs because the algorithm is based on them. It is explained how to represent graphs in order to understand the algorithm's input.
Afterwards, it is explained how the algorithm works its runtime complexity is described. An example of implementation written in Python pseudo-code helps to illustrate the content.
Finally, multifaceted scopes of application are presented and show why dijkstra's algorithm is absolutely necessary in our modern age.
 			% Kurzfassung Deutsch/English
\tableofcontents 						% Inhaltsverzeichnis
%--------------------------------------------------------------------------
\mainmatter                        		% Hauptteil (ab hier arab. Seitenzahlen)
%--------------------------------------------------------------------------
% Die Kapitel werden in separaten .tex-Dateien abgelegt und hier eingebunden.
\chapter{Einleitung und Problemstellung}

EINLEITUNG
\chapter{Graphen}

\section{Definition Graph}


-Für Dijkstra auf gerichtete Graphen ausgelegt.

\begin{theorem}
Ein \textbf{Graph} G besteht aus einer Menge X [deren Elemente Knotenpunkte genannt werden] und einer Menge U, wobei jedem Element u $\in$ U in eindeutiger Weise ein geordnetes oder ungeordnetes Paar von [nicht notwendig verschiedenen] Knotenpunkten, x,y $\in$ X zugeordnet ist.
Ist jedem u $\in$ U ein geordnetes Paar von Knoten zugeordnet, so heißt der Graf \textbf{gerichtet}, und wir schreiben 
	$G= (X, U)$.
Die Elemente von U werden in diesem Fall als \textbf{Bögen} bezeichnet.

Ist jedem u $\in$ U ein ungeordnetes Paar von Knotenpunkten zugeordnet, so heißt der Graph \textbf{ungerichtet} und wir schreiben 
	$G=[X,U]$. 
Die Elemente von U bezeichnen wir dann als \textbf{Kanten.}
\end{theorem} \footnote{Bieß, Graphentheorie, S.9}


Ein gerichteter Graph besteht somit aus einer Menge aus geordneten Knotenpunkten. Dadurch, das diese Punkte geordnet sind, ist eine Verlaufsrichtung festgelegt, in welcher der Graph zeigt.
Somit kann sich eine geometrische Figur ergeben.


[Grafik hier einfügen]


Wenn man diese Punkte als Richtungen nun als Straßennetz sähe, würde sich durch die Punkte ein Straßenverlauf abbilden mit unterschiedlichen Verlaufsrichtungen.
Somit verbünde nicht jede Strecke direkt jeden Punkt. Um jenen kürzesten Weg zu finden, der zwei bestimmte Punkte miteinander verbindet, kann man den Dijkstra-Algorithmus anwenden, welcher im Folgenden erklärt wird.



\section{Datenstrukturen zur Repräsentation von Graphen}

\emph{-Speicherung in Adjazenzmatrix}
-Graph G=(V,E) wird in einer booleschen m*m Matrix gespeichert ($m = \sharp V$)
 mit 1 falls (i,j) element E (Kantenmenge) und 0 falls nicht
 [formale Def einfügen]
 -Nachteil: erfordert unverhältnismäßig viel Speicher, wenn der Graph nur wenig    	  Kanten hat
 -Abhilfe: Zusatzmatrix, die nur bedeutsame Einträge speichert
 -insgesamt aber trotzdem ineffizient bei Graphen mit wenig Kanten
 
 [Grafik]
 
\emph{-Speicherung in Adjazenzlisten}
-für jeden Knoten wird eine lineare, verkettete Liste seiner ausgehenden Kanten gespeichert
-Die Knoten werden als lineares Feld gespeichert (d.h. jeder Knoten im Feld zeigt auf eine Liste)
-effizienter als Adjazenzmatrix, weil kein Speicherplatz verschwendet wird

[Grafik]

\emph{-Speicherung in doppelt verketteten Listen}
-jedes Element enthält Zeiger auf die beiden Nachbarelemente sowie auf eine Kantenliste (wie bei Adjazenzliste, s.o.)
-diese Darstellung besitzt die den Adjazenzlisten fehlende Dynamik, ist aber natürlich komplizierter

[Grafik]

Quelle:
Algorithmen und Datenstrukturen, T.Ottmann/P.Widmayer, Kap. 8 Graphenalgorithmen, S.539-544

GRAPHEN
\chapter{Dijktra - Algorithmus}

\section{Erklärung}

-löst das Shortest-Path Problem (oder Single-Source Shortest Path Problem)
[Def Problem einfügen]
-Problemstellung: Wie finde ich den kürzesten Pfad von meinem Startknoten s zu einem Zielknoten u ? Mit kurz ist nicht die Länge gemeint, sondern die Minimierung der Pfadkosten!
-Eingabe sind Digraphen (ungerichtete Graphen lassen sich leicht in Digraphen umwandeln, indem man zu jeder Kante (u,v) eine Kante (v,u) mit gleichen Kosten hinzufügt)
-Annahme: für jeden Knoten existiert auch ein Pfad
-funktioniert nach dem Entwurfsprinzip Greedy

\underline{kurze Erklärung zu Greedy-Algorithmen:}
-greedy = engl. gierig
- zu Beginn  ist Teillösung T = $\emptyset$
-im nächsten Schritt werden alle möglichen Erweiterungen in Betracht gezogen und nach ihrem Nutzen bewertet. Zur Bewertung wird ein Greedy-Kriterium formuliert.
-Die Erweiterung, die den maximalen Nutzen bringt, wird ausgewählt und die Lösung damit erweitert. 
- Eine einmal durchgeführte Erweiterung wird nicht mehr rückgängig gemacht! Deshalb sind Greedy-Algorithmen sehr effizient.
-Die durch sukzessive Erweiterung erzielte Lösung ist optimal.

Dijkstra 
-Grundidee:  Pfadkosten d(u) für jeden Knoten u schätzen aus einer Menge Q (alle restlichen Knoten). Nach jeder Erweiterung werden die Schätzungen aktualisiert

-genaueres Vorgehen:
1. Initalisierung: Man beginnt mit Q = V, d.h. alle Knoten sind enthalten, d(s) = 0 (Kosten des Startknoten auf 0 setzen). Alle anderen d(u) werden auf undefiniert gesetzt, weil deren Schätzwerte noch nicht bekannt sind.
2.Erweiterung: Jetzt wird die Greedy-Auswahlregel verwendet, d.h. man sucht in Q nach dem Knoten mit dem kleinsten Schätzwert und entfernt diesen aus Q. (Streng genommen wird die Lösung nicht erweitert, da die Lösungsmenge nicht explizit konstruiert wird. T entspricht V/Q)
3.Aktualisierung: Nachfolger von v durchgehen und deren Schätzwerte d(u) ggf. aktualisieren, falls diese nun schneller erreichbar sind.

-wenn man zu jedem Knoten u den Vorgänger p(u) = v in einer Liste speichert, lässt sich so zu jedem Knoten der kürzeste Pfad finden 
(s, ... , p(p(u)),p(u),u) 


\section{Komplexität}

-\textbf{O(m²) }bei schlechter Implementierung
-bei Implementierung mit Priority-Queue in geeigneter Datenstruktur (z.B. heapq in Python) lässt sich  eine deutlich bessere Laufzeit von \textbf{O(k*log m)} erreichen [k = Kantenzahl des Graphen]

\section{Implementierung}

\lstset{language=Python}
\begin{lstlisting}
from heapq import heappush, heappop

def dijkstra_pq(G,s):
    m = len(G)                              #O(1)
    pq = []       #priority queue           #O(1)
    d = [None]*m  #kosten                   #O(m)
    p = [None]*m  #vorgaenger		       #O(m)
    d[s] = 0                                #O(1)
    heappush(pq, (0,s))                     #O(m)
    while pq:                               #O(m)
        (_,v) = heappop(pq)                 #O(log m)
        for u in G[v]:                      #O(deg(v)) --> O(k)  
            alt = d[v] + G[v][u]            #O(1)
            if d[u]== None or alt < d[u]:   #O(1)
                d[u] = alt		            #O(1)	
               	p[u] = v                    #O(1)
                heappush(pq, (alt,u))       #O(log m)        
    return d,p

def shortest_path(s,v,p):
	if v == None:
		return []
	else:
		return shortest_path(s,p[v],p) + [v]
 
#Graph: 
    
def define_G():
    G = [   {1:1, 4:4, 2:2},   # Nachfolger von s
            {3:3, 4:1},        # von u
            {4:2, 5:3},        # von x
            {},                # von y
            {3:1, 5:2},        # von v
            {}                 # von z
        ]
    return G


G = define_G()
d,p = dijkstra_pq(G,0)
print( shortest_path(0,5,p))
print( shortest_path(0,3,p))
\end{lstlisting}

\section{Anwendungsbereiche}
Der Dijkstra-Algorithmus wird noch heute oft verwendet und bietet einen großen Nutzen beispielsweise im Verkehrswesen.
So wird er oft für die Routenplanung von Warentransporten verwendet, aber auch für die Streckenberechnungen von Zügen und Flugzeugen.
Hier bedeutet jeder Umweg höhere Transportkosten, da Treibstoff und Fahrtzeit steigen.

Doch auch in Kommunikationsnetzwerken findet der Weg-Findungs-Algorithmus von Dijkstra noch immer Verwendung.
Zwischen Rechnernetzen findet Routing statt. Hier kosten Umwege wertvolle Kapazitäten, welche sehr begrenzt sind. Somit ist es sinnvoll, die kürzesten Kommunikationswege zu finden, damit die Kapazitäten nicht unnötig belastet werden.


-Informationsnetzwerke (Anzahl Links Domain Suchmaschine)

DIJKTRA - ALGORITHMUS
\chapter{Zusammenfassung}

ZUSAMMENFASSUNG + AUSBLLICK
%\chapter{Zusammenfassung und Ausblick}

In diesem Kapitel soll die Arbeit noch einmal kurz zusammengefasst werden. Insbesondere sollen die wesentlichen Ergebnisse Ihrer Arbeit herausgehoben werden. Erfahrungen, die z.B. Benutzer mit der Mensch-Maschine-Schnittstelle gemacht haben oder Ergebnisse von Leistungsmessungen sollen an dieser Stelle pr�sentiert werden. Sie k�nnen in diesem Kapitel auch die Ergebnisse oder das Arbeitsumfeld Ihrer Arbeit kritisch bewerten. W�nschenswerte Erweiterungen sollen als Hinweise auf weiterf�hrende Arbeiten erw�hnt werden.
% ...
%--------------------------------------------------------------------------
\backmatter                        		% Anhang
%-------------------------------------------------------------------------
\bibliographystyle{geralpha}			% Literaturverzeichnis
\bibliography{literatur}     			% BibTeX-File literatur.bib
%--------------------------------------------------------------------------
\printindex 							% Index (optional)
%--------------------------------------------------------------------------
\begin{appendix}						% Anh�nge sind i.d.R. optional
   	\chapter{Glossar}

\abbreviation{DisASTer}		{DisASTer (Distributed Algorithms Simulation Terrain), A platform for the Implementation of Distributed Algorithms}
\abbreviation{DSM}			{Distributed Shared Memory}
\abbreviation{AC}			{Linearisierbarkeit (atomic consistency)}
\abbreviation{SC}			{Sequentielle Konsistenz (sequential consistency)}
\abbreviation{WC}			{Schwache Konsistenz (weak consistency)}
\abbreviation{RC}			{Freigabekonsistenz (release consistency)}
			% Glossar   
\end{appendix}

\end{document}
