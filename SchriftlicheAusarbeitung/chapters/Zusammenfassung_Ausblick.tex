\chapter{Ausblick}


Es hat sich herausgestellt, dass der Dijkstra-Algorithmus trotz seiner Bedeutung von überschaubarer Komplexität ist und daher leicht verständlich ist.
Dijkstras ursprüngliche Implementierung wurde im Laufe der Zeit im Hinblick auf Laufzeit und Codelänge bzw.  Codekomplexität optimiert. Es ist abzusehen, dass auch in Zukunft noch Optimierungen stattfinden werden.
Dabei ist das Grundprinzip jedoch immer gleich geblieben. Die Bedeutung und Gegenwart des Algorithmus ist uns zwar oft nicht bewusst, dennoch ist er aus dem modernen Alltag nicht mehr wegzudenken.
Überall, wo Transportwege und Kommunikationsnetze im Spiel sind, ist Dijkstra gefragt. Wer will schon ewig warten, bis seine Ware geliefert wird? Oder bis eine Internetseite lädt? Man stelle sich einmal vor, wie langsam das Internet wäre, wenn die kürzesten Pfade nicht so schnell gefunden würden.
Alles in allem lässt sich sagen, dass der Algorithmus gerade in der modernen Zeit, in der es oft auf Geschwindigkeit und Effizienz ankommt, unabdingbar ist.
