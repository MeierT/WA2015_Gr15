\chapter{Ausblick}
ALT
Routenplanung Algorithmen sind eine der Algorithmen, ohne die die Welt wie wir sie kennen nicht existieren könnte. Sie sind eine der Grundbausteine auf der die heutige Effizienz der Menschheit aufbaut! \\ 
Diese Effizienz kann man heutzutage nicht mehr vernachlässigen, es ist etwas das die heutige Menschheit ausmacht. \\
\\
Dijkstra macht mit seinem Algorithmus unser Leben effizienter, stressfreier, erlebenswerter. Denn wer will schon lange auf die Verbindung zu einem Server warten, oder mit dem Auto einen Umweg fahren? \\
Die Vorteile, welche der Dijkstra-Algorithmus der Menschheit bietet, sind für uns schon längst Alltag. Sie sind für moderne Menschen essenziell und dies wird sich in der Zukunft auch nichtmehr ändern. Algorithmen welche das "Kürzeste Wege Problem"lösen können werden immer gebraucht werden. \\
Daher werden sich auch in der Zukunft Leute mit diesem Problem befassen und den Dijstra-Algorithmus benutzen bzw. möglicher weise auch einen neuen Algorithmus entwickeln.
\\
\\
NEU
Es hat sich herausgestellt, dass der Dijkstra-Algorithmus trotz seiner Bedeutung von überschaubarer Komplexität ist und daher leicht verständlich ist.
Dijkstras ursprüngliche Implementierung wurde im Laufe der Zeit im Hinblick auf Laufzeit und Codelänge bzw.  Codekomplexität optimiert. Es ist abzusehen, dass auch in Zukunft noch Optimierungen stattfinden werden.
Dabei ist das Grundprinzip jedoch immer gleich geblieben. Die Bedeutung und Gegenwart des Algorithmus ist uns zwar oft nicht bewusst, dennoch ist er aus dem modernen Alltag nicht mehr wegzudenken.
Überall, wo Transportwege und Kommunikationsnetze im Spiel sind, ist Dijkstra gefragt. Wer will schon ewig warten, bis seine Ware geliefert wird? Oder bis eine Internetseite lädt? Man stelle sich einmal vor, wie langsam das Internet wäre, wenn die kürzesten Pfade nicht so schnell gefunden würden.
Alles in allem lässt sich sagen, dass der Algorithmus gerade in der modernen Zeit, in der es oft auf Geschwindigkeit und Effizienz ankommt, unabdingbar ist.
