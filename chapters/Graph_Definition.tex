\chapter{Graphen}

\section{Definition Graph}



\begin{theorem}
Ein \textbf{Graph} G besteht aus einer Menge X [deren Elemente Knotenpunkte genannt werden] und einer Menge U, wobei jedem Element u $\in$ U in eindeutiger Weise ein geordnetes oder ungeordnetes Paar von [nicht notwendig verschiedenen] Knotenpunkten, x,y $\in$ X zugeordnet ist.
Ist jedem u $\in$ U ein geordnetes Paar von Knoten zugeordnet, so heißt der Graf \textbf{gerichtet}, und wir schreiben 
	$G= (X, U)$.
Die Elemente von U werden in diesem Fall als \textbf{Bögen} bezeichnet.

Ist jedem u $\in$ U ein ungeordnetes Paar von Knotenpunkten zugeordnet, so heißt der Graph \textbf{ungerichtet} und wir schreiben 
	$G=[X,U]$. 
Die Elemente von U bezeichnen wir dann als \textbf{Kanten.}
\end{theorem} \footnote{Bieß, Graphentheorie, S.9}\\

[Grafik hier einfügen]


Ein gerichteter Graph, auch Digraph genannt,  besteht somit aus einer Menge aus geordneten Knotenpunkten \footnote{Ottman, Datenstrukturen und Algorithmen, S590}. Dadurch, dass diese Punkte geordnet sind, also zu jedem Element U genau ein Paar Knotenpunkte zugeordnet wird, ist eine Verlaufsrichtung festgelegt, in welcher der Graph zeigt.

Die Verbindungslinien zwischen zwei Knotenpunkten werden Kanten genannt, welchen Kosten zugeordnet sind. Diese Kosten sind im Vorneherein festgelegte, nicht negative Werte. Sie sind  nicht mit der Kantenlänge zu verwechseln.

Somit kann sich eine geometrische Figur ergeben.\\

Weiterhin gibt es auch ungerichtete Graphen, welche jedoch in dieser Ausarbeitung keine Bedeutung haben und somit an dieser Stelle nicht weiter erläutert werden.
In der folgenden Arbeit wird jedoch mit anderen Bezeichnungen gearbeitet. Anstelle von der von Gieß gebrauchten Bezeichnung G=(X,U), wird G=(V,E) verwendet.







