\chapter{Graphen}

\section{Definition Graph}

\section{Datenstrukturen zur Repräsentation von Graphen}

\emph{-Speicherung in Adjazenzmatrix}
-Graph G=(V,E) wird in einer booleschen m*m Matrix gespeichert (m = \sharp V)
 mit 1 falls (i,j) element E (Kantenmenge) und 0 falls nicht
 [formale Def einfügen]
 -Nachteil: erfordert unverhältnismäßig viel Speicher, wenn der Graph nur wenig    	  Kanten hat
 -Abhilfe: Zusatzmatrix, die nur bedeutsame Einträge speichert
 -insgesamt aber trotzdem ineffizient bei Graphen mit wenig Kanten
 
 [Grafik]
 
\emph{-Speicherung in Adjazenzlisten}
-für jeden Knoten wird eine lineare, verkettete Liste seiner ausgehenden Kanten gespeichert
-Die Knoten werden als lineares Feld gespeichert (d.h. jeder Knoten im Feld zeigt auf eine Liste)
-effizienter als Adjazenzmatrix, weil kein Speicherplatz verschwendet wird

[Grafik]

\emph{-Speicherung in doppelt verketteten Listen}
-jedes Element enthält Zeiger auf die beiden Nachbarelemente sowie auf eine Kantenliste (wie bei Adjazenzliste, s.o.)
-diese Darstellung besitzt die den Adjazenzlisten fehlende Dynamik, ist aber natürlich komplizierter

[Grafik]

Quelle:
Algorithmen und Datenstrukturen, T.Ottmann/P.Widmayer, Kap. 8 Graphenalgorithmen, S.539-544

GRAPHEN