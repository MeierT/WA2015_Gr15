\chapter{Einleitung und Problemstellung}

Wir leben in einer modernen, schnelllebigen Welt. Die Globalisierung hat jeden Kontinent verbunden und sowohl Marktwirtschaft als auch Tourismus florieren mehr denn je. 
Doch in solch einer Welt gibt es auch viele logistische Herausforderungen. Waren sollen ausgeliefert werden, Passagierflugzeuge müssen mit passenden Routen ausgestattet werden, Touristen wollen mit dem Auto oder per Bahn an ihre Urlaubsziele kommen. \\
Doch wie ist ein solcher Aufwand am effizientesten zu bewältigen? 
Hierfür hat der Mathematiker Dijkstra bereits in den Fünfziger Jahren einen Algorithmus entwickelt, der noch heute in der Praxis Anwendung findet. \\
In dieser Arbeit geht es um genau diesen Algorithmus. 
Dafür müssen jedoch vorher Begriffe wie \glqq Graphen \grqq erklärt werden und Datenstrukturen für deren Repräsentation festgelegt werden.
Des Weiteren wird auf den Algorithmus direkt eingegangen, mit einer einführenden Erklärung, der Klärung der Komplexität und anschließender Implementierung in Python Pseudo-Code.
Abschließend werden aktuelle Anwendungsbeispiele präsentiert, um zu zeigen, dass dieses Verfahren zwar schon vor längerer Zeit entwickelt wurde, doch an Relevanz nichts eingebüßt hat.
\\
\\
