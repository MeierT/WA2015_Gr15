\chapter{Einleitung und Problemstellung}

In der folgenden Arbeit geht es um die Vorstellung des Dijkstra-Algorithmus.
Hierzu wird im Folgenden grob erklärt, was Graphen sind und in welchen Datenstrukturen sie repräsentiert werden können.
Danach wird auf den Algorithmus direkt eingegangen, mit einer einführenden Erklärung, der Klärung der Komplexität und anschließender Implementierung in Python Pseudo-Code.
Zum Schluss wird noch auf aktuelle Anwendungsbeispiele des Algorithmus eingegangen, um zu zeigen, dass dieses Verfahren zwar schon vor längerer Zeit entwickelt wurde, doch immer noch in der Praxis relevant ist.
