\chapter{Dijkstra-Algorithmus}

\section{Erklärung}

-Algorithmus zur Weg-Findung

Der Dijkstra-Algorithmus ist ein Verfahren zur Berechnung des Kürzesten Weges zu einem bestimmten Zielpunkt auf der Basis eines gegebenen Ausgangspunktes und einem Netz aus möglichen Wegen.
So kann ein Netz aus möglichen Wegen beispielsweise ein Straßennetz sein.
Ausgangspunkt ist jener Punkt, auf dem man sich gerade befindet und der Zielpunkt soll das Fahrtziel sein.
Der Algorithmus sucht nun nach dem kürzesten Weg um dieses Ziel zu erreichen.

Entwickelt wurde dieses Verfahren vom Mathematiker Edsker W. Dijkstra (1930 - 2002), einen niederländischen Professor für Mathematik an der technischen Hochschule Eindhoven, im Jahre 1959.



-Weiteres Vorgehen:
	- Genauer beschreiben, wie der Algorithmus funktioniert.
		-> Nach Ottman/Wiedmayer S. 406 ff.
		
