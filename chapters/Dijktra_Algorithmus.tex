\chapter{Dijktra - Algorithmus}

\section{Erklärung}

-löst das Shortest-Path Problem (oder Single-Source Shortest Path Problem)
[Def Problem einfügen]
-Problemstellung: Wie finde ich den kürzesten Pfad von meinem Startknoten s zu einem Zielknoten u ? Mit kurz ist nicht die Länge gemeint, sondern die Minimierung der Pfadkosten!
-Eingabe sind Digraphen (ungerichtete Graphen lassen sich leicht in Digraphen umwandeln, indem man zu jeder Kante (u,v) eine Kante (v,u) mit gleichen Kosten hinzufügt)
-Annahme: für jeden Knoten existiert auch ein Pfad
-funktioniert nach dem Entwurfsprinzip Greedy

\underline{kurze Erklärung zu Greedy-Algorithmen:}
-greedy = engl. gierig
- zu Beginn  ist Teillösung T = \emptyset
-im nächsten Schritt werden alle möglichen Erweiterungen in Betracht gezogen und nach ihrem Nutzen bewertet. Zur Bewertung wird ein Greedy-Kriterium formuliert.
-Die Erweiterung, die den maximalen Nutzen bringt, wird ausgewählt und die Lösung damit erweitert. 
- Eine einmal durchgeführte Erweiterung wird nicht mehr rückgängig gemacht! Deshalb sind Greedy-Algorithmen sehr effizient.
-Die durch sukzessive Erweiterung erzielte Lösung ist optimal.

Dijkstra 
-Grundidee:  Pfadkosten d(u) für jeden Knoten u schätzen aus einer Menge Q (alle restlichen Knoten). Nach jeder Erweiterung werden die Schätzungen aktualisiert

-genaueres Vorgehen:
1. Initalisierung: Man beginnt mit Q = V, d.h. alle Knoten sind enthalten, d(s) = 0 (Kosten des Startknoten auf 0 setzen). Alle anderen d(u) werden auf undefiniert gesetzt, weil deren Schätzwerte noch nicht bekannt sind.
2.Erweiterung: Jetzt wird die Greedy-Auswahlregel verwendet, d.h. man sucht in Q nach dem Knoten mit dem kleinsten Schätzwert und entfernt diesen aus Q. (Streng genommen wird die Lösung nicht erweitert, da die Lösungsmenge nicht explizit konstruiert wird. T entspricht V/Q)
3.Aktualisierung: Nachfolger von v durchgehen und deren Schätzwerte d(u) ggf. aktualisieren, falls diese nun schneller erreichbar sind.

-wenn man zu jedem Knoten u den Vorgänger p(u) = v in einer Liste speichert, lässt sich so zu jedem Knoten der kürzeste Pfad finden 
(s, ... , p(p(u)),p(u),u) 


\section{Komplexität}

-\textbf{O(m²) }bei schlechter Implementierung
-bei Implementierung mit Priority-Queue in geeigneter Datenstruktur (z.B. heapq in Python) lässt sich  eine deutlich bessere Laufzeit von \textbf{O(k*log m)} erreichen [k = Kantenzahl des Graphen]

\section{Implementierung}

from heapq import heappush, heappop

def dijkstra_pq(G,s):
    m = len(G)                              #O(1)
    pq = []       #priority queue           #O(1)
    d = [None]*m  #kosten                   #O(m)
    p = [None]*m  #vorgaenger		       #O(m)
    d[s] = 0                                #O(1)
    heappush(pq, (0,s))                     #O(m)
    while pq:                               #O(m)
        (_,v) = heappop(pq)                 #O(log m)
        for u in G[v]:                      #O(deg(v)) --> O(k)  
            alt = d[v] + G[v][u]            #O(1)
            if d[u]== None or alt < d[u]:   #O(1)
                d[u] = alt		            #O(1)	
               	p[u] = v                    #O(1)
                heappush(pq, (alt,u))       #O(log m)        
    return d,p

def shortest_path(s,v,p):
	if v == None:
		return []
	else:
		return shortest_path(s,p[v],p) + [v]
 
#Graph: 
    
def define_G():
    G = [   {1:1, 4:4, 2:2},   # Nachfolger von s
            {3:3, 4:1},        # von u
            {4:2, 5:3},        # von x
            {},                # von y
            {3:1, 5:2},        # von v
            {}                 # von z
        ]
    return G


G = define_G()
d,p = dijkstra_pq(G,0)
print( shortest_path(0,5,p))
print( shortest_path(0,3,p))

\section{Anwendungsbereiche}

-Verkehrsnetzwerke (Routenplanung für Warentransport, Luftfahrt, Zug, Kosten sind hierbei z.B. Reisezeiten, Entfernung und Maut

-Kommunikationsnetzwerke (Routing Rechnernetze, Kosten = Kapazitäten, Übertragungszeiten)

-Informationsnetzwerke (Anzahl Links Domain Suchmaschine)

DIJKTRA - ALGORITHMUS