\section{Dijkstra}


\begin{frame}
	\begin{block}<1->{white node}
		Ein ``white node'' ist ein Knoten über den noch nichts bekannt ist.
	\end{block}
	\begin{block}<2->{grey node}
		Ein ``grey node'' ist ein Knoten zu dem bereits ein Weg gefunden wurde. Ob dies der ideale Weg ist kann man allerdings noch nicht sagen.
	\end{block}
	\begin{block}<3->{black node}
		Ein ``black node'' ist ein Knoten zu dem bereits der optimale Weg gefunden wurde.
	\end{block}
\end{frame}

\begin{frame}
	\begin{block}{Berechnung von Knotenwerten}
		Der Schätzwert $k_A$ eines Knotens $K_A$ kann berechnet werden, wenn es einen Knoten $K_V$ gibt, dessen Schätzwert $k_V$ bekannt ist und eine gerichtete Kante von $K_V$ nach $K_A$ mit bekannten Gewichtung $g$ existiert.
	\end{block}
	\vfill
	Berechnung: $k_A = k_V + g$
\end{frame}

\begin{frame}
	\begin{block}{Anfang}
		Der Startknoten wird als ``black node'' mit dem Wert 0 definiert. 
		Alle seine Nachfolgerknotenwerte werden bestimmt. Diese werden zu ``grey node''
	\end{block}
\end{frame}
\begin{frame}
	\begin{block}{Schritt}
		Der ``grey node'', welcher den niedrigsten Wert besitzt, wird zu einem ``black nodes''.
		Bei allen seinen Nachfolgern wird deren Wert von dem aktuell ausgewählten ``grey node'' neu berechnet. Wenn der Wert kleiner als der Wert des Knoten ist, so wird dieser Wert überschrieben.
		Knoten, welche das erste mal einen Wert zugeordnet bekommen, sind jetzt ``grey nodes''.

		relaxieren benennen.
	\end{block}
\end{frame}
\begin{frame}
	\begin{block}{Ende}
		Wenn der Zielknoten ein ``black node'' wird, hat man den idealen Weg vom Startknoten und Endknoten gefunden.
	\end{block}

\end{frame}