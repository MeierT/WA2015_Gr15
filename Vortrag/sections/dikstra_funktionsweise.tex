\begin{frame}
	\begin{block}<1->{white note}
		Ein ``white note'' ist ein Knoten über den noch nichts bekannt ist.
	\end{block}
	\begin{block}<2->{grey note}
		Ein ``grey note'' ist ein Knoten zu dem bereits ein Weg gefunden wurde. Ob dies der ideale Weg ist kann man allerdings noch nicht sagen.
	\end{block}
	\begin{block}<3->{black note}
		Ein ``black note'' ist ein Knoten zu dem bereits der ideale Weg gefunden wurde.
	\end{block}
\end{frame}

\begin{frame}
	\begin{block}{Berechnung von Knotenwerten}
		Der Knotenwert $k_A$ eines Knotens $K_A$ so kann berechnet werden, wenn es einen Knoten $K_V$ gibt dessen Knotenwert $k_V$ bekannt ist und eine gerichtete Kante von $K_V$ nach $K_A$ mit bekannten Kosten $p$ existiert.
	\end{block}
	\vfill
	Berechnung: $k_A = k_V + p$
\end{frame}

\begin{frame}
	\begin{block}{Anfang}
		Der Startknoten wird als ``black note'' mit dem Wert 0 definiert. 
		Alle seine Nachfolgerknotenwerte werden bestimmt. Diese werden zu ``grey notes''
	\end{block}
\end{frame}
\begin{frame}
	\begin{block}{Schritt}
		Der ``grey note'', welcher den niedrigsten Wert besitzt, wird zu einem ``black notes''.
		Bei allen seinen Nachfolgern wird deren Wert neu von dem aktuell ausgewählten ``grey note'' berechnet. Wenn der Wert kleiner als der Wert des Knoten ist, so wird dieser Wert überschrieben.
		Knoten, welche das erste mal einen Wert zugeordnet bekommen, sind jetzt ``grey notes''.
	\end{block}
\end{frame}
\begin{frame}
	\begin{block}{Ende}
		Wenn der Zielknoten ein ``black note'' wird, hat man den idealen Weg vom Startknoten und Endknoten gefunden.
	\end{block}

\end{frame}